\documentclass[]{article}
\usepackage{polski}
\usepackage[utf8]{inputenc}

%opening
\title{Edytor Wielokątów z wypełnianiem i obcinaniem - dokumentacja}
\author{Szymon Adach}

\begin{document}

\maketitle

\section{Polecenia}
\begin{enumerate}
	\item \textbf{Dodanie wielokąta} - PPM\footnote{Prawy Przycisk Myszy} $\rightarrow$ Add polygon \newline
	Program przechodzi w tryb rysowania. Do momentu zakończenia rysowania (poprzez zamknięcie wielokąta, klikając na pierwszy, oznaczony na niebiesko, wierzchołek)) lub wciśnięcia przycisku Esc okno znajduje się w trybie rysowania.
	\item \textbf{Tryb wspomagania relacji przy rysowaniu} - klawisz F12 w trybie rysowania włącza lub wyłącza tryb wspomagania relacji (o aktualnym stanie użytkownik jest powiadamiany MessageBoxem). Zalecane jest ustawienie tej opcji przed rozpoczęciem rysowania wielokąta.
	\item \textbf{Usunięcie wielokąta} - PPM $\rightarrow$ Remove polygon.
	Do momentu wybrania usuwanego wielokąta lub wciśnięcia przycisku Esc okno znajduje się w trybie usuwania.
	\item \textbf{Przejście do trybu przesuwania wielokąta} - PPM $\rightarrow$ Move polygon. \newline
	Aby przesunąć wielokąt, należy kliknąć LPM\footnote{Lewy Przycisk Myszy} na dowolny jego wierzchołek. Wielokąt będzie przesuwany do momentu kolejnego kliknięcia LPM lub wciśnięcia przycisku Esc.
	\item \textbf{Przejście do trybu wyznaczania części wspólnej wielokątów} - PPM $\rightarrow$ Polygon intersection. \newline
	Należy wybrać dwa wielokąty, który przecięcie wyznaczamy, poprzez kliknięcie na ich krawędź bądź wierzchołek. Zaznaczone są one czerwonym kolorem. Na ekranie pozostaje wyznaczona część wspólna (o ile istnieje), a uprzednio zaznaczone wielokąty są usuwane. Z trybu można wyjść wciskając klawisz ESC.
	\item \textbf{Dodawanie wierzchołka w środku wybranej krawędzi} - kliknięcie LPM na wybraną krawędź.
	\item \textbf{Usunięcie wierzchołka} - kliknięcie na wybranym wierzchołku LPM trzymając jednocześnie przycisk Ctrl. Usunięcie wierzchołka w trójkącie powoduje usunięcie całego trójkąta (nie dopuszczamy istnienia pojedynczych odcinków na bitmapie).
	\item \textbf{Przesuwanie wierzchołka} - kliknięcie LPM na wybranym wierzchołku uaktywnia stan przesuwania wierzchołka. Jest on przesuwany aż do kolejnego kliknięcia lub wciśnięcia Esc.
	\item \textbf{Menu boczne} - opcje w menu bocznym są samotłumaczące :). Po kliknięciu na przyciski możemy  zmienić kolory bądź wczytać nową bitmapę z pliku.
\end{enumerate}
\end{document}