\documentclass[]{article}
\usepackage{polski}
\usepackage[utf8]{inputenc}

%opening
\title{Edytor Wielokątów - dokumentacja}
\author{Szymon Adach}

\begin{document}

\maketitle

\section{Polecenia}
\begin{enumerate}
	\item \textbf{Dodanie wielokąta} - PPM\footnote{Prawy Przycisk Myszy} $\rightarrow$ Add polygon \newline
	Do momentu zakończenia rysowania lub wciśnięcia przycisku Esc okno znajduje się w trybie rysowania.
	\item \textbf{Usunięcie wielokąta} - PPM $\rightarrow$ Remove polygon
	Do momentu wybrania usuwanego wielokąta lub wciśnięcia przycisku Esc okno znajduje się w trybie usuwania.
	\item \textbf{Przejscie do trybu przesuwania wielokąta} - PPM $\rightarrow$ Move polygon. \newline
	Aby przesunąć wielokąt, należy kliknąć LPM\footnote{Lewy Przycisk Myszy} na dowolną jego krawędź lub wierzchołek. Wielokąt będzie przesuwany do momentu kolejnego kliknięcia.
	\item \textbf{Dodawanie wierzchołka w środku wybranej krawędzi} - kliknięcie LPM na wybraną krawędź.
	\item \textbf{Usunięcie wierzchołka} - kliknięcie na wybranym wierzchołku LPM trzymając jednocześnie przycisk Ctrl.
	\item \textbf{Przesuwanie wierzchołka} - kliknięcie LPM na wybranym wierzchołku uaktywnia stan przesuwania wierzchołka. Jest on przesuwany aż do kolejnego kliknięcia.
	\item \textbf{Dodanie relacji} - kliknięcie PPM na wybranej krawędzi a następnie wybór jednego z typów relacji (gdzie None oznacza brak relacji dla krawędzi). W przypadku wybrania relacji długości wyswietla się okno dialogowe, pozwalające wpisać oczekiwaną długość.
\end{enumerate}
\section{Algorytmy zastosowane do relacji}
Relacje są propagowane. Wierzchołki mają stosy. Fajnie jest wtedy. TODO.
\end{document}
