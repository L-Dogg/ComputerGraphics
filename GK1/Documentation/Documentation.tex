\documentclass[]{article}
\usepackage{polski}
\usepackage[utf8]{inputenc}

%opening
\title{Edytor Wielokątów - dokumentacja}
\author{Szymon Adach}

\begin{document}

\maketitle

\section{Polecenia}
\begin{enumerate}
	\item \textbf{Dodanie wielokąta} - PPM\footnote{Prawy Przycisk Myszy} $\rightarrow$ Add polygon \newline
	Program przechodzi w tryb rysowania. Do momentu zakończenia rysowania (poprzez zamknięcie wielokąta, klikając na pierwszy, oznaczony na niebiesko, wierzchołek)) lub wciśnięcia przycisku Esc okno znajduje się w trybie rysowania.
	\item \textbf{Tryb wspomagania relacji przy rysowaniu} - klawisz F12 w trybie rysowania włącza lub wyłącza tryb wspomagania relacji (o aktualnym stanie użytkownik jest powiadamiany MessageBoxem). Zalecane jest ustawienie tej opcji przed rozpoczęciem rysowania wielokąta.
	\item \textbf{Usunięcie wielokąta} - PPM $\rightarrow$ Remove polygon.
	Do momentu wybrania usuwanego wielokąta lub wciśnięcia przycisku Esc okno znajduje się w trybie usuwania.
	\item \textbf{Przejście do trybu przesuwania wielokąta} - PPM $\rightarrow$ Move polygon. \newline
	Aby przesunąć wielokąt, należy kliknąć LPM\footnote{Lewy Przycisk Myszy} na dowolny jego wierzchołek. Wielokąt będzie przesuwany do momentu kolejnego kliknięcia LPM lub wciśnięcia przycisku Esc.
	\item \textbf{Dodawanie wierzchołka w środku wybranej krawędzi} - kliknięcie LPM na wybraną krawędź.
	\item \textbf{Usunięcie wierzchołka} - kliknięcie na wybranym wierzchołku LPM trzymając jednocześnie przycisk Ctrl. Usunięcie wierzchołka w trójkącie powoduje usunięcie całego trójkąta (nie dopuszczamy istnienia pojedynczych odcinków na bitmapie).
	\item \textbf{Przesuwanie wierzchołka} - kliknięcie LPM na wybranym wierzchołku uaktywnia stan przesuwania wierzchołka. Jest on przesuwany aż do kolejnego kliknięcia lub wciśnięcia Esc.
	\item \textbf{Dodanie relacji} - kliknięcie PPM na wybranej krawędzi a następnie wybór jednego z typów relacji (gdzie None oznacza brak relacji dla krawędzi). W przypadku wybrania relacji długości wyświetla się okno dialogowe, pozwalające wpisać oczekiwaną długość.
\end{enumerate}
\section{Algorytmy zastosowane do relacji}
Zastosowano algorytm propagowania relacji w przód i w tył przy pomocy iterowania po liście krawędzi. Przed próbą zastosowania relacji dla krawędzi, zapisywane są obecne stany wierzchołków. Jeżeli zarówno nowe, jak i wszystkie dotychczasowe ograniczenia mogą być zastosowane, to zapisywany jest nowy stan wielokąta. W przeciwnym wypadku przywracany jest stan sprzed prób aplikowania nowej relacji.
\end{document}
